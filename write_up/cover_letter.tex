\documentclass[12pt]{letter}
\usepackage[margin=2.54cm, a4paper]{geometry}
\signature{\vspace*{-1cm} 
	Paul Dolder \\ 
	Dr C\'oil\'in Minto\\ 
	Prof Jean-Marc Guarini \\
	Dr Jan Jaap Poos}
\address{Paul Dolder \\
	Marine and Freshwater Research Centre \\ 
	Galway-Mayo Institute of Technology \\ 
	Dublin Road \\ 
	Galway \\ 
	Ireland}

\begin{document}

\begin{letter}{Prof. Brian D. Fath \\ Department of Biological Sciences \\
		Towson University \\ 8000 York Road \\ Towson \\ Maryland 21252 \\ USA}
\opening{Dear Professor Fath,}

We would like to submit an \emph{Original research} paper titled \emph{`Highly
	resolved spatiotemporal simulations for exploring mixed fisheries
	dynamics'} for consideration in \emph{Ecological modelling}. In this
paper we develop a simulation model (\emph{MixFishSim}) of a ecological-fishery
system where individual fishing vessels exploit multiple heterogeneously
distributed fish populations with full spatiotemporal population dynamics.
Using\emph{MixFishSim} we address critically important sampling issues which as
the importance of spatial and temporal scale when modelling fishery
interactions, particularly in relation to modelling observational data from a
system with preferential sampling.\\

Our approach incorporates, in an emergent manner, how fishers exploit a dynamic
natural resource with uncertain knowledge of its distribution. We detail how
this affects our understanding of the fisheries interactions with multiple fish
populations, which cannot be achieved with conventional modelling approaches.
By capturing dynamics observed in fisheries data in a novel simulation model
where the entire dynamics of the system (including the true spatial
distribution of the populations) are known, we expose that degrading the
spatial and temporal resolution of data through aggregation reduces our ability
to define effective spatial management measures. This would not of been
possible without the highly details and resolved simulation approach used here.
The framework has many potential additional applications (e.g. monitoring
survey design, index standardisation for fisheries assessment, in-year fishery
and biological modelling, testing adaptive management approaches, comparison of
heuristic and mechanistic fishery effort dynamics models among others).
MixFishSim is made available as a documented R software package for users to
explore.\\

Should you consider our work suitable for review, we would like to suggest the
following potential reviewers:

\begin{itemize} 
	\item \textbf{Dr Tom Carruthers:} Expert in data limited fisheries and
		spatial fisheries models. Assistant Professor, University of British
		Columba, Institute for the Ocean and Fisheries, Room 335, 2202
		Main Mall, Vancouver, British
		Columbia, Canada. Email: t.carruthers@oceans.ubc.ca.\\ 
	\item \textbf{Dr Coby Needle:} Expert in spatial fisheries modelling.
		Marine Scotland, Marine Laboratory 375 Victoria Road, Aberdeen,
		AB11 9DB, Scotland, UK. Email: coby.needle@gov.scot. \\   
	\item \textbf{Prof. Richard Bailey:} Expert in coupled
		human-environment interactions. St Catherine's College, Oxford
		University, Manor Road, Oxford, OX1 3UJ. Email:
		richard.bailey@ouce.ox.ac.uk. 
	\item \textbf{Prof. Edward Codling:} Expert in animal movement
		behaviour and marine fisheries systems. STEM 5.11, University
		of Essex, Colchester Campus, Wivenhoe Park, Colchester, CO4
		3SQ, UK. Email:ecodling@essex.ac.uk.  
\end{itemize}

Thank you for your time and consideration.

\closing{Yours Sincerely,} \end{letter} 

\end{document}
