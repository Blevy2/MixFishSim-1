\documentclass{article}


\usepackage[a4paper, margin=1in]{geometry}
\usepackage[utf8]{inputenc}
\usepackage{lscape}
\usepackage{longtable}
\usepackage{hhline}
\usepackage{booktabs}

\begin{document}

\large

Address here: cover letter on other computer. \\
\\
Dear xyz, \\
RE: Ms. No. ECOMOD-19-631 \\
\\
Thank you for your email of 5 December 2019 with the favourable 'minor
revisions` review of our manuscript submission. We have reviewed all the
comments made by the editor and the two reviewers and addressed them
accordingly. We thank the reviewers for a constructive review which undoubtedly
improves the manuscript and we found little reason to disagree with their
comments. A summary of our changes are presented overleaf.  \\

We look forward to hearing back positively on these changes, and publication of
the manuscript in Ecological Modelling. \\
\\
Kind Regards,\\
\\
Paul J. Dolder, Cóilín Minto, Jean-Marc Guarini \& Jan Jaap Poos

\newpage

\begin{landscape}

\underline{Editor's comments} \\
\\
\textit{
After considering the comments of the two reviewers, which are provided below,
I think your manuscript should be acceptable for publication after satisfactory
revision.  Please address the reviewers' comments when revising your
manuscript.  Thank you again for submitting your manuscript to Ecological
Modelling.
} \\

Thank you for your comments, we have provided a detailed response to each of
the points made by the reviewers; we hope this satisfies your requirements and
look forward to seeing the manuscript published. \\


\underline{Reviewer 1 comments} \\

\textit{This paper introduces a simulation framework for assessing the effects
	of different spatial and temporal observation scales on management of
	mixed fisheries. An application of the framework is presented, and the
	findings from the example application provided insights into the
	importance of data resolution on design and effectiveness of closed
	areas. The model code is included as an R package so that other
	researchers will be able to implement it readily to address a wide
	variety of management questions. The question of appropriate scales is
	an important one, and I think this will be a valuable tool for
	researchers and managers. I recommend that it be published after minor
	revisions.} \\

Thank etc..

\large
	\begin{center}
	\begin{longtable}{p{12cm} | p{12cm}}
%	\caption{Highly resolved spatiotemporal simulations for exploring mixed
%	fishery dynamics}
		\toprule
		Comment & Response \\
		\\
		\hline
		\\
		\textbf{Abstract} & \\
\\
I think it would be helpful to mention the name of the R package (MixFishSim)
that accompanies this paper somewhere in the abstract. People will search for
the package name, it would be nice if the term was included in the abstract so
the paper will show up in the search too. & Thank you, we agree and have
included in the first sentence of the abstract. \\
\\
"true underlying populations" - somehow make sure readers know this is
referring to simulated populations. & We have changed to "true (simulated)
underlying populations" \\
\\
Could add something like "bycatch avoidance", or something like that, somewhere
in key words or abstract; people searching for that term would be glad to find
this paper, I would think! & \\
\\
I think the abstract could be streamlined a bit. You seem to have 2 major
objectives with this paper: 1) introducing and describing the simulation
framework that could answer a number of important management questions, and 2)
demonstrating its usefulness with an example application. One full paragraph
for each should be adequate, with a conclusion that states the usefulness of
the framework for answering a number of fisheries management questions.  & \\
\\
		\hline
		\textbf{Introduction} &  \\
\\
Nice introduction to the need for the simulation framework & Thank you, we have
kept this text unchanged. \\
\\
Line 62: "Do different sources of sampling-derived fisheries data reflect"… &
Thanks, corrected. \\
\\
		hline
		\textbf{Methods} &  \\
\\
Why was the Matern covariance structure chosen? I am just curious about this;
does it reflect the spatial autocorrelation of animal distributions better than
a Gaussian or other type of covariance structure? & We used the Matérn
covariance structure as it allows for a non-linear decrease in correlation as
distance increases and is flexible, in that it reverts to an exponential form
under specific conditions. As such it is capable of taking any number of forms,
as required. Alternative covariance structures could also be used, but Matérn
is the most commonly used. We've added text to say "We use the most commonly
used Matérn covariance structure as it is a flexible form that under certain
conditions is of the same form as an exponential function and it enables us to
model the spatial autocorrelation observed in animal populations where density
is more similar in nearby locations, but that correlation decreases
non-linearly...." \\ 
\\
It would be helpful to have explanation of thermal tolerance earlier than line
155 & We have now added a sentence at the beginning of the section that reads
"[described by a set of probabilities] which are affected by the suitability of
habitat, temperature in a cell and the thermal tolerance of a population to
that temperature." \\
\\
The paragraph starting at line 179 would be helpful to have at the beginning of
the section & Agree, we have moved this to now be the second paragraph in the
section. \\
\\
Line 187, make tenses consistent (e.g. determined vs. determines) - also this
sentence is a little awkward to read. & We have changed so that it reads in the
past tense and rewritten the paragraph to improve readability. \\
\\
		\hline
		\textbf{Results} &  \\
\\
Line 376: What does "Visualized using Gerritsen (2014)" mean? & \\
\\
Line 397: Are these the same figures in Table 7? If so, just refer to the table
and summarize & \\
\\
Line 400: move \% sign & This has been corrected. [Note, may be redundant if
change text] \\
\\
Line 406: "(in red)" - leave reference to color for figure legend & Removed
reference to color. \\
\\
Line 410: "Again" doesn't need to be there, same in line 413 & Agree, it has
been removed. \\
\\
Line 416: This section is a little confusing and needs to be refined/streamlined. & \\
\\
Line 418: This paragraph should be in methods. But you did it for all
populations, right? Not just population 3? & \\
\\
		\hline
		\textbf{Discussion / Conclusion} &  \\
\\
The discussion and conclusion overlap a bit; I felt I was reading similar
things over again; perhaps these can be streamlined? & \\
\\
Line 560: change "reduced" to "reduce" & Thank you, corrected. \\
\\
Line 663: change "hypothesis" to "hypotheses" & Done. \\
\\
		\hline
		\textbf{Figures} &  \\
\\
Figure 1: nicely summarizes the model. It took me a little bit to figure out
that "Rec" referred to recruitment - even though it was in the legend (I missed
it when I read it the first time)… is it possible to choose an abbreviation
that makes it more obvious, maybe something like "Recruit" or  even "Recr"? It
also occurs to me that there is an opportunity to introduce the terminology
from your R package here to make it very clear to the people using your code
how the functions work together… this is only an idea though… or maybe a
similar figure could be included with the vignette. & \\
\\
Figure 5: Do the colors indicate the dominant population in each cell? It is
hard to imagine that one would assume only the presence of specific population
in the cell? Or do I have the wrong idea about what the plot is showing? & \\
\\
Figure 9: So the top two panels show spatial distribution for population 3?
That needs to be added to the legend. Also there should be a red box on the top
plot too. & \\
\\
Figure 10: I really like this plot! Very effective. & \\
		\bottomrule
	\end{longtable}

\end{center}

\underline{Reviewer 2 comments} \\

\textit{The paper Dolder et al. presents a new highly resolved spatiotemporal
	model that simulates populations of species and fishery dynamics to
	assess if data from commercial catches and fixed-site sampling surveys
	on different spatiotemporal scales represent the real distribution of
	the populations, and if these data are effective to base  management
	programs. Moreover, they presents other applications of the model.} \\

\textit{This is a very interesting paper, and the model provides several
	applications for management programs. I have only few greater questions
	and some small issues, most of them about the presentation of the
	information.}  \\

Thank etc..


\begin{center}

\large

	\begin{longtable}{p{12cm} | p{12cm}}
%	\caption{Highly resolved spatiotemporal simulations for exploring mixed
%	fishery dynamics}
		\toprule
		Comment & Response \\
		\\
		\hline
		\\
	\textbf{General comments} &  \\
\\
		 How many species does the model simulate? The authors refer to
		 the issue of catching unwanted species (vulnerable or low
		 quota species) when available quota species is caught, and it
		 suggested me that populations of different species is modeled
		 at the same time. However, they  did not describe how many
		 species could be simulated in each simulation, only that four
		 populations were simulated. Does the populations simulated in
		 different simulations, or all of them are simulated
		 simultaneously? The results indicate that the four populations
		 were simulated in the same simulation. Please, clarify this in
		 the MM.  &  \\
\\
There are too many figures and they should be better edited. Why the name of
the row and columns are inside the graphics? The axes scales and the axes title
should be greater. I would also send the figure 2, 3 and 9 as supplementary
data.  & \\
\\
The legends of the supplementary figures should be more informative. In the
Figure S2, what the light gray spots are? Habitat preference? Are the squares
the spawning areas? In the figure S3, are each square a week?  & \\
\\
I am not native English, but the language seems good for me. However, I missed
several commas along the text that make the sentences too long and hard to
understand. I suggest a revision of this.  & \\

		\hline

	\textbf{Small issues} &  \\
\\
Line 14 - 18: This phrase is too long and hard to understand. Please, rephrase
this in shorter sentences to clarify this idea. & \\
\\
Lines 20 - 43: These two paragraphs seem redundant.  & \\
\\
Line 68 - 72: The authors mentioned that the MixFishSim could be used to infer
if fisheries data are the real community structure. I suggest adding one or two
phrases to explain how the model could help in this. & \\
\\
Line 90. The authors presented the time-step of each module, except the
Recruitment dynamics. The recruitment dynamics probably follow the population
dynamic, but it would be interesting present this explicitly. & \\
\\
Line 282: How these population parameters were selected? Randomly? Using real
data? Or the authors only created them? & \\
\\
Line 426: What does adapted means in this context? Exploring new opened areas?
& \\
\\
Line 613: Do not use parenthesis inside of other parenthesis. & We have changed
to ", e.g." from "( e.g." to eliminate one set of parenthesis. \\
\\
Table 2 and table 3 are not cited in the text Citation of the supplementary
figures are out of order & \\
\\
Figure 3: Individual years ARE the light grey lines    & Thanks! Corrected. \\
\\
Figure 4. What do the purple spots represent? Describe these spots in the
capture Figure 7. Add what " res"  means in the caption.  & \\
\\
Figure 8. What F means? & \\
\\
		\bottomrule
	\end{longtable}

\end{center}

\end{landscape}

\end{document}
